\documentclass[8pt]{extarticle}
\usepackage[most]{tcolorbox}
\usepackage{xcolor,soul,enumitem,tocloft,amsmath,amssymb,amsthm,amsfonts,graphicx,hyperref,mathtools,setspace,fancyhdr,multicol,braket}
\usepackage[utf8]{inputenc}
\usepackage[margin=1in]{geometry}
\usepackage[version=4]{mhchem}

\setlength{\headheight}{15pt}

% EM DASH: —
% DEGREE SYMBOL: °

\begin{document}
  \newgeometry{left=0.25in, right=0.25in, top=1in, bottom=1in}
  \pagestyle{fancy}
  \fancyhead[C]{ScalarFit Functional Forms (v1.0.0)}
  \begin{center}
    \textbf{IN ALL CASES, THE GUESS VECTOR IS DEFINED AS $[c_1, c_2, c_3,...,c_n]$ AS REQUIRED BY THE FUNCTIONAL FORM.}
  \end{center}
  \vspace{30pt}
  \begin{multicols}{3}
  \section*{Cubic}
  $$y=c_1x^3+c_2x^2+c_3x+c_4$$
  \section*{Exponential}
  $$y=c_1\exp(c_2x)+c_3$$
  \section*{Gaussian}
  $$y=c_1\exp\left(-c_2(x-c_3)^2\right)+c_4$$
  \section*{Linear}
  $$y=c_1x+c_2$$
  \section*{Logarithmic}
  $$c_1\ln(x-c_2)+c_3$$
  \section*{Logistic}
  $$y=\frac{c_1c_2\exp(c_3x)}{c_1-c_2+c_2\exp(c_3x)}$$
  \section*{Polynomial}
  $$y=\sum_{i=1}^{n}{c_ix^{i-1}}$$
  This functional form does not enforce a specific guess size. Instead, it fits the data to a polynomial with degree $n-1$ where $n$ is the length of the guess vector.
  \section*{Quadratic}
  $$y=c_1(x-c_2)^2+c_3$$
  \section*{Rational}
  $$y=\frac{\sum_{i=1}^{m}{c_ix^{i-1}}}{\sum_{i=m+1}^{m+n}{c_ix^{i-(m+1)}}}$$
  The case of the rational functional form is treated as the ratio of two polynomial functional forms. The guess vector should have length $m+n$. The degree of the polynomial in the numerator is $m-1$. The degree of polynomial in the denominator is $n-1$.
  \section*{Sinusoid}
  $$y=c_1\sin(c_2x-c_3)+c_4$$
  \end{multicols}
\end{document}